%%%%%%%%%%%%%%%%%%%%%%%%%%%%%%%%%%%%%%%%%%%%%%%%%%%%%%%%%%%%%%%%%%%%%
%% PREAMBLE

\documentclass[a4paper, 12pt]{article}

% General document formatting
\usepackage[margin=1.25in]{geometry}
\usepackage[parfill]{parskip}
\usepackage[utf8]{inputenc}

% Figures
\usepackage{graphicx}
\usepackage[section]{placeins} 
% References 
\usepackage[authoryear,round]{natbib}
    
% Related to math
\usepackage{amsmath,amssymb,amsfonts,amsthm}

% Hyperlinks
\usepackage[colorlinks]{hyperref}

% Author details
\title{Noise-Busters: How to Sharpen Your Monte Carlo}
\author{06009246}
\date{Compiled: \today}
%%%%%%%%%%%%%%%%%%%%%%%%%%%%%%%%%%%%%%%%%%%%%%%%%%%%%%%%%%%%%%%%%%%%%%

\begin{document}

\maketitle

\textbf{Github Repo:} \href{https://github.com/kkquest/mc-tutorial-R}{\color{blue}{LINK-TO-THE-TAGGED-RELEASE}}

\section{Project Description}

This project, \textbf{Noise-Busters: How to Sharpen Your Monte Carlo}, 
offers a hands-on tutorial in R that unpacks three fundamental variance-reduction 
strategies — antithetic sampling, control variates and importance sampling — 
using the estimation of the rare-event probability \(P\bigl(Z_1 + Z_2 > 4\bigr), \quad Z_i \sim \mathcal{N}(0,1)\) as a running example. 

The tutorial consists of a suite of standalone, well-organised scripts. 
In particular:

\begin{enumerate}
  \item \texttt{simulations.R}: defines the baseline sampler and auxiliary simulators;
  \item \texttt{variance\_reduction\_methods.R}: implements each reduction algorithm;
  \item \texttt{confidence\_intervals.R}: constructs data frames of estimates with confidence bounds;
  \item \texttt{plotting.R}: produces clear, annotated ggplot2 visualisations;
  \item \texttt{true\_value.R}: supplies the analytical probability for reference.
\end{enumerate}

Each script is modular, immediately executable and fully documented 
with roxygen2-style comments. The tutorial itself is authored as a Quarto notebook, 
weaving explanatory text, reproducible code and publication-quality plots 
to demonstrate the $O(1/\sqrt{n})$ convergence bottleneck of naïve Monte Carlo 
and to compare how each technique accelerates convergence and curtails sampling variability.

\textbf{A strong emphasis is placed on reproducibility and sound software engineering: 
simulations are seeded for exact replication, computationally intensive chunks are cached, 
and the environment is locked down with renv.} Rather than embedding the R functions 
implementing each variance-reduction technique directly in the narrative, 
the tutorial focuses on conceptual exposition and visual demonstration; 
interested readers are most welcome to explore the complete codebase and 
accompanying materials in the project’s GitHub repository.


%=============================================
\pagebreak
%=============================================

\section{Assessment Criteria}

\textbf{Technical Competence:}
\begin{itemize}
  \item Developed a modular R codebase (in \texttt{R/}) implementing standard Monte Carlo 
  and three variance‐reduction methods.
  \item Instrumented convergence diagnostics via \texttt{confidence\_intervals.R} 
  to compute and tabulate mean ± CI over increasing sample sizes.
\end{itemize}

\textbf{User Interface:} 
\begin{itemize}
  \item Authored a Quarto notebook with a clear, engaging structure.
  \item Encouraged curious readers to inspect every standalone R script via the GitHub repository, 
  fostering hands‐on exploration of the utilised functions.
  \end{itemize}

\textbf{Analysis and Interpretation:} 
\begin{itemize}
  \item Framed each section with precise questions, 
  contrasted empirical estimates with the closed‐form true value (\texttt{true\_estimate}), 
  and interpreted CI widths in relative terms.
  \item Compared bias, variance, and compute time across methods — to
  highlight not only the varience precision improvement of the discussed methods 
  but also their computational efficiency.
\end{itemize}

\textbf{Presentation and Communication:} 
\begin{itemize}
  \item Followed a narrative arc (“Hook → Chaos → Methods → Comparison → Takeaways”) with short text blocks, 
  call-outs, and transition headings to maintain reader engagement.
  \item Produced simple but informative ggplot2 visualisations in \texttt{plotting.R}, 
  annotated with true-value lines and concise axis labels, ensuring immediate interpretability.
\end{itemize}

\textbf{Reproducibility and Documentation:} 
\begin{itemize}
  \item Managed package versions with \texttt{renv}, 
  seeded all random draws for exact replication, and cached expensive chunks 
  to guarantee consistent outputs across renders.
  \item Documented every function with roxygen2‐style comments (in \texttt{R/*.R}), 
  and provided a one-click Quickstart in \texttt{README.md} to restore the environment 
  and render the notebook.
\end{itemize}

\textbf{Project Management:} 
\begin{itemize}
  \item Organised the repository into logical folders; create and followed  
  a milestone plan with concrete deadlines.
  \item Captured the finished tutorial in a single comprehensive commit and 
  marked the submission milestone with a  Git tag: \texttt{v1.0.1}.
\end{itemize}

%=============================================
\pagebreak
%=============================================

\section{Project Reflection}

\textbf{Learnings:}

\begin{itemize}
    \item I discovered how orchestrating modular scripts—from samplers to plotting — 
    creates a flexible framework that can be extended to new Monte Carlo problems with minimal friction.
    \item Creating convergence diagnostics and CI visualisations reinforced the 
    practical importance of tracking error bounds as a function of $n$, 
    rather than relying on point estimates alone.
    \item Embedding roxygen2 documentation and renv‐driven reproducibility early on 
    made the codebase feel like a polished software library rather than a one-off analysis notebook.
\end{itemize}


\textbf{Challenges:}

\begin{itemize}
    \item Initially, I scoped the tutorial as three parts—introducing simple Monte Carlo 
    via the $\pi$ example, detailing variance-reduction techniques, 
    and applying them to a real-world problem — but quickly realised this plan was too ambitious 
    and refocused on the core rare-event narrative for depth over breadth.
    \item I did not allocate any time to writing \texttt{testthat} tests, 
    leaving key simulation functions unverified; I now appreciate that formal tests 
    are crucial for catching regressions and ensuring statistical correctness, 
    and would include comprehensive test coverage from day one in any future project.
\end{itemize}

\textbf{Further Development:}

\begin{itemize}
    \item Going forward, I am keen to explore interactive Shiny or Voila widgets
    that let users experiment with sample sizes and variance-reduction parameters in real time, 
    turning static plots into dynamic learning tools.
\end{itemize}

\end{document}
