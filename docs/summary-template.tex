%%%%%%%%%%%%%%%%%%%%%%%%%%%%%%%%%%%%%%%%%%%%%%%%%%%%%%%%%%%%%%%%%%%%%
%% PREAMBLE

\documentclass[a4paper, 12pt]{article}

% General document formatting
\usepackage[margin=1.25in]{geometry}
\usepackage[parfill]{parskip}
\usepackage[utf8]{inputenc}

% Figures
\usepackage{graphicx}
\usepackage[section]{placeins} 
% References 
\usepackage[authoryear,round]{natbib}
    
% Related to math
\usepackage{amsmath,amssymb,amsfonts,amsthm}

% Hyperlinks
\usepackage[colorlinks]{hyperref}

% Author details
\title{YOUR PROJECT TITLE}
\author{YOURCID}
\date{Compiled: \today}
%%%%%%%%%%%%%%%%%%%%%%%%%%%%%%%%%%%%%%%%%%%%%%%%%%%%%%%%%%%%%%%%%%%%%%

\begin{document}

\maketitle

\textbf{Github Repo:} \href{https://github.com/zakvarty/eds-notes-quarto/releases/tag/v1.2.0-alpha}{\color{blue}{REPLACE-WITH-LINK-TO-YOUR-TAGGED-RELEASE}}

\section{Project Description (approx. 250 words)}


Provide a short project description of 200-300 words here. Lorem ipsum dolor sit amet, consectetur adipiscing elit. Curabitur aliquam turpis tellus, non semper dui tincidunt vel. Nam non mauris nec sem aliquam eleifend ac tincidunt nisl. Sed interdum leo vel mi condimentum, et convallis felis consequat. Fusce commodo erat urna, in imperdiet ante aliquet eu. Maecenas tempus accumsan libero, et consectetur velit varius eu. Vivamus a augue nibh. Maecenas et dignissim erat, quis consectetur est. Ut scelerisque metus ipsum, eu hendrerit purus consequat vel. Praesent sed condimentum ipsum. Fusce non eros at tortor gravida ornare varius vel mauris. In vehicula libero a ligula euismod feugiat.

Donec semper egestas cursus. Morbi a facilisis libero, at lacinia velit. Maecenas mattis tincidunt mauris a venenatis. Vestibulum urna odio, tincidunt quis nibh nec, porttitor porttitor elit. Morbi efficitur ultricies urna, auctor porta dui vehicula et. Pellentesque elementum viverra metus nec efficitur. In dolor magna, auctor nec dapibus quis, tristique ut neque. Integer consectetur vel tortor vel dapibus. Nunc malesuada sem eu accumsan volutpat. Morbi a ullamcorper quam. Vestibulum ante ipsum primis in faucibus orci luctus et ultrices posuere cubilia curae; Aenean a quam ac diam congue rutrum. Integer ultrices congue mauris eget mollis. Nullam vel venenatis massa.

Vestibulum dapibus, nulla vel rhoncus iaculis, purus ligula faucibus nibh, at porta felis tellus ac orci. Suspendisse pretium, metus sit amet molestie lobortis, nisi nibh tincidunt magna, quis tempus elit risus at ante. Morbi non ante elit. 

%=============================================
\pagebreak
%=============================================

\section{Assessment Criteria}

\textit{In 2-3 bullet points each and at most 1 page in total, describe how your submission addresses each of the assessment criteria below. You may delete this italicised text when filling in the template.} 

\textbf{Technical Competence:} Proficiency in data collection, processing, analysis, and coding.

\begin{itemize}
    \item Item 1
    \item Item 2
\end{itemize}

\textbf{User Interface:} Design, functionality, and usability of the final data product.

\begin{itemize}
    \item Item 1
    \item Item 2
\end{itemize}

\textbf{Analysis and Interpretation:} Depth of analysis, appropriate use of statistical methods, and meaningful interpretation.

\begin{itemize}
    \item Item 1
    \item Item 2
\end{itemize}

\textbf{Presentation and Communication:} Clarity, organisation and effectiveness of written and visual communication.

\begin{itemize}
    \item Item 1
    \item Item 2
\end{itemize}

\textbf{Reproducibility and Documentation:} Clarity and completeness of documentation for product use and reproducibility.

\begin{itemize}
    \item Item 1
    \item Item 2
\end{itemize}

\textbf{Project Management:} Considered and effective use of project management and version control systems.

\begin{itemize}
    \item Item 1
    \item Item 2
\end{itemize}

%=============================================
\pagebreak
%=============================================

\section{Project Reflection}

\textit{Reflect on the experience of creating your data product. In 6 bullet points and at most 1 page total, summarise the following.} 

\begin{itemize}
    \item \textit{3 things you have learned as part of this process,}
    \item \textit{2 aspects of the project that you found challenging or would approach differently with hindsight,} 
    \item \textit{1 aspect of the project that you would like to learn more about in the future.}
\end{itemize}

\textit{You may delete this italicised text when filling in the template.} 

\textbf{Learnings:}

\begin{itemize}
    \item Item 1
    \item Item 2
    \item Item 3
\end{itemize}

\textbf{Challenges:}

\begin{itemize}
    \item Item 1
    \item Item 2
\end{itemize}

\textbf{Further Development:}

\begin{itemize}
    \item Item 1
\end{itemize}

\end{document}
